\section{Podsumowanie}

\subsection{Rezultaty}
Link do zdalnego repozytorium GitHub: \href{https://github.com/Sawoboh/Informatyka_Projek_1.git}{https://github.com/Sawoboh/Informatyka\_{}Projek\_{}1.git} \\
Znajduje sie na nim pliki o nazwie:
\begin{itemize}
\item Transformacje\_{}Projekt.py - główny plik w którym znajdują się transformacje oraz wywołanie przykładowego pliku txt wraz z zapisym do do pliku o nazwie Wyniki transformacji.txt.
\item wsp\_{}inp.txt - zawierają przykładowe dane, które możemy przliczyć. 
\item Kalkulator.py -  Cztery plik importujący biblioteke argparse. Za pomocą wiersza poleceń (cmd) należy  podać dane do obliczeń.
\item Testy\_{}dla\_{}funkcji.py - plik importujący biblioteke pytest. Za pomocą wiersza poleceń (cmd) podać dane do obliczenia.
\item TKinter.py - plik wywołuje kalkulator graficzny i okno wynikowe. 
\end{itemize}

\subsection{Umiejętności nabyte}
\begin{itemize}
	\item Sprawne pisanie plików tekstowych w latex w celu nauki skorzystliśmy z książki \citep{Borkowski.Przybylski2015}
	\item Pisanie kodu obiektowego w Pythonie 
	\item Posługiwanie się bibliotekami takimi jak: argparse, pytest, tkinter, os, numpy \citep{argparse} \citep{argparse2}
	\item Praca zespołowa z wykorzystaniem platformy Github
	\item Poprawienie jakości i przyśpieszenie pisania kodu w Pythonie
\end{itemize}

\subsection{Spostrzerzenia, probelmy i ich rozwizania:}
Spostrzerzenia:
\begin{itemize}
\item Wraz ze wzrostem ilości czasu poświęconego na projekt, zauważano coraz wiecej luk w funkcjach, które usprawniono.
\item Cały czas program nie jest kompletny w 100\%. Zawsze się znajdzie nowy pomysł który można zaimplementować.
\end{itemize}

\begin{table}[h!]
	\centering
	\begin{tabular}{|p{7cm}|p{7cm}|}
		\hline
		 Problem  &  Rozwizanie \\
		\hline
		Brak spójności w długości liczb w tabeli przez co plik wynikowy nie wyglądałby schludnie & Dodanie funkcji na dodanie spacji w stringu opisany w rozdziale 3.7  \\ \hline
		Pojawianie się błedu w funkcji PL2000 i PL 1992 dla ($\phi$, $\lambda$) nie leżącego na terenie Polski & Dodane warunku na ($\phi$, $\lambda$) w którym jeśli warunek nie jest spełniony zapisuje w notatniku myslinik (-)\\ \hline
		Dopisywanie do istniejącego pliku wyników & Wykorzystano biblioteke os do rozpoznania czy dany plik istnieje i podanie odpowiednich komend w przypadku gdy plik zostaje stworzony i do pliku zostają dopisane wyniki\\ \hline
		Nieumijętność korzystania z biblioteki argparse & Doptytanie się na zajęciach informatyki geodezyjnej prowadzącego o wytłumaczenie jak korzystać z biblioteki \\ \hline
		Nieumijętność korzystania z portala Github & Skorzystanie z materiałów uspodtępnionych przez prowadzącego zajecia informatyka geodezyjna\\ \hline
		Brak możliości zresetowania tablicy z wynikami w pliku pytest &  \\ \hline
		
	\end{tabular}
\end{table}