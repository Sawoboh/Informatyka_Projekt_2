\section{Podsumowanie}

\subsection{Rezultaty}
Link do zdalnego repozytorium GitHub: \href{https://github.com/Sawoboh/Informatyka_Projekt_2.git}{https://github.com/Sawoboh/Informatyka\_Projekt\_2.git} w gałęz master \\
Znajduje sie na nim pliki o nazwie:
\begin{itemize}
	\item folder wtyczka\_projekt\_3 - folder z naszą wtyczką stworzoną w tym zadaniu.
	\item folder nasz\_latex - folder ze sprawozdaniem
	\item ttt.qgz - plik w qgis na której można testować wtyczke
\end{itemize}

\subsection{Umiejętności nabyte}
\begin{itemize}
	\item Sprawne pisanie plików tekstowych w latex w celu nauki skorzystliśmy z książki \citep{Borkowski.Przybylski2015}
	\item Umiejętność tworzenia wtyczek w Qgis 
	\item Tworzenie wtyczek w Qt Designer 
	\item Praca zespołowa z wykorzystaniem platformy Github
	\item Poprawienie jakości i przyśpieszenie pisania kodu w Pythonie
\end{itemize}

\subsection{Spostrzerzenia, probelmy i ich rozwizania:}
Spostrzerzenia:
\begin{itemize}
	\item Wraz ze wzrostem ilości czasu poświęconego na projekt, zauważano coraz wiecej luk w funkcjach, które usprawniono.
	\item Cały czas program nie jest kompletny w 100\%. Zawsze się znajdzie nowy pomysł który można zaimplementować.
	\item Projekt dostarczył nam dużo wiedzy która przyda nam się w przyszłości.
\end{itemize}

\begin{table}[h!]
	\centering
	\begin{tabular}{|p{7cm}|p{7cm}|}
		\hline
		\textbf{Problem}  &  \textbf{Rozwizanie} \\
		\hline
		Brak uporządkowanego wybierania punktów przez Qgis (dziwne poligon i pola powierzchni) & Wykorzystanie algorytmu do uporządkowanie położenia punktów zgodnie z ruchem wskazówek zegara   \\ \hline
		Brak umiejętnosci korzystania z Radio Button & Wykorzystanie QComboBox  \\ \hline
		Współrzedne nie mają podanej wysokości & Stworzenie własnej warstyw za pomocą wcztania pliku z XYh \\ \hline
		Zapis pliku do folderu z qgis & Postanowiono ze plik będzie zawsze zpisywał się na pulpicie użytkownika \\ \hline
		
	\end{tabular}
\end{table}