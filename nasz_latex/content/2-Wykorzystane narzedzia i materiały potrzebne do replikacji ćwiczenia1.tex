\section{Wykorzystane narzedzia i materiały potrzebne do replikacji ćwiczenia}

\subsection{Wybrany język programowania i interpreter Spyder}

\begin{itemize}
	\item Python - język programowania, w którym napisany jest skrypt ćwiczenia.
	\item Spyder - jest to środowisko programistyczne dla języka Python, które zawiera edytor kodu, interpreter, konsolę i wiele innych funkcjonalności.
	\item Najlepiej pobrać Spydera za pośrednictwem anacondy, która ma domyślie zainstalowane środowisko programistyczne Spyder \href{https://www.anaconda.com/download}{www.anaconda.com/download}	\citep{Anaconda}
	\item Qgis - program do który jest napisana nasza wtyczka. Zapośrednitwem tego programu zliczamy punkty itp.
\end{itemize}

\subsection{System operacyjny}

Ten skrypt został napisany w systemie operacyjnym Microsoft (Windows 10 oraz Windows 11).

\subsection{Potrzebne biblioteki i pliki}

Do wykonania ćwiczenia należy użyć następujących bibliotek:
\begin{enumerate}
	\item os - to biblioteka standardowa w języku python, która zapewnia interfejs do operacji na systemie operacyjnym (np. dostęp do plików, zarządzanie procesami, zmiana katalogu roboczego)
	\item PyQt5 - PyQt5 to zestaw narzędzi do tworzenia aplikacji graficznych w języku Python przy użyciu biblioteki Qt. Qt jest popularnym frameworkiem do tworzenia interfejsów użytkownika. Nie jest on wbudowany w pythona dlatego trzeba go zainstalować.
	\item qgis.PyQt - moduł w bibliotece QGIS Python, który jest wykorzystywany do tworzenia interfejsów użytkownika w QGIS przy użyciu PyQt. Nie jest to samodzielna biblioteka PyQt5. Działa w połączeniu z biblioteką QGIS.
	\item qgis.core - to moduł w bibliotece QGIS Python, który dostarcza podstawowe funkcje i klasy do pracy z danymi przestrzennymi w QGIS. Nie jest to samodzielna biblioteka PyQt5. Działa w połączeniu z biblioteką QGIS.
	\item PyQt5.QtCore - to moduł w bibliotece PyQt5, który zawiera podstawowe klasy i funkcje rdzenne dla PyQt5. Moduł ten dostarcza elementy i funkcjonalności, które są częścią modułu QtCore w oryginalnej bibliotece Qt.
	\item math - to biblioteka pythona, dzięki której można wykonywać podstawowe operacje matematyczne.
\end{enumerate}

Należy również pobrać plik o nazwie "ttt.qgz", który znajduje się na zdalnym repozytorium GitHub pod linkiem: \href{https://github.com/Sawoboh/Informatyka_Projekt_2.git}{https://github.com/Sawoboh/Informatyka\_Projekt\_2.git} da on możliwość testowania wtyczki w qgis na podanych punktach. Pilki "Przykladowe\_wspolrzedne.txt" oraz\\ "Przykladowe\_wspolrzedne2.txt" zawierają przykłądowe plik tekstowe które można wczytać do wtyczki. Plikiy jest to zapis warstwy, która znajduje się w pliku "ttt.qgz". 