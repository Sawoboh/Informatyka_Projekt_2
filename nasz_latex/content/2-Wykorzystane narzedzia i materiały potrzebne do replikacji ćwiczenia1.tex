\section{Wykorzystane narzedzia i materiały potrzebne do replikacji ćwiczenia}

\subsection{Wybrany język programowania i interpreter Spyder}

\begin{itemize}
\item Python - język programowania, w którym napisany jest skrypt ćwiczenia.
\item Spyder - jest to środowisko programistyczne dla języka Python, które zawiera edytor kodu, interpreter, konsolę i wiele innych funkcjonalności.
\item Najlepiej pobrać Spydera za pośrednictwem anacondy, która ma domyślie zainstalowane środowisko programistyczne Spyder \href{https://www.anaconda.com/download}{www.anaconda.com/download}	\citep{Anaconda}
\item Qgis - program do który jest napisana nasza wtyczka. Zapośrednitwem tego programu zliczamy punkty itp.
\end{itemize}

\subsection{System operacyjny}

Ten skrypt został napisany w systemie operacyjnym Microsoft (Windows 10 oraz Windows 11).

\subsection{Potrzebne biblioteki i pliki}

Do wykonania ćwiczenia należy użyć następujących bibliotek:
\begin{enumerate}


\end{enumerate}

Należy również pobrać plik o nazwie "ttt.qgz", który znajduje się na zdalnym repozytorium GitHub pod linkiem: \href{https://github.com/Sawoboh/Informatyka_Projekt_2.git}{https://github.com/Sawoboh/Informatyka\_Projekt\_2.git} da on możliwość testowania wtyczki w qgis.  