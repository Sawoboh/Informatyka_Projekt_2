\section{Wykorzystane narzedzia i materiały potrzebne do replikacji ćwiczenia}

\subsection{Wybrany język programowania i interpreter Spyder}

\begin{itemize}
\item Python - język programowania, w którym napisany jest skrypt ćwiczenia.
\item Spyder - jest to środowisko programistyczne dla języka Python, które zawiera edytor kodu, interpreter, konsolę i wiele innych funkcjonalności.
\item Najlepiej pobrać Spydera za pośrednictwem anacondy, która ma domyślie zainstalowane środowisko programistyczne Spyder \href{https://www.anaconda.com/download}{www.anaconda.com/download}	\citep{Anaconda}
\end{itemize}

\subsection{System operacyjny}

Ten skrypt został napisany w systemie operacyjnym Microsoft (Windows 10 oraz Windows 11).

\subsection{Potrzebne biblioteki i pliki}

Do wykonania ćwiczenia należy użyć następujących bibliotek:
\begin{enumerate}
\item Numpy - to biblioteka w języku Python, która służy do obliczeń numerycznych i analizy danych. Numpy dostarcza wiele narzędzi do pracy z wielowymiarowymi tablicami danych oraz narzędzi do wykonywania operacji matematycznych i statystycznych na tych tablicach. Numpy nie jest wbudowany w Pythona, ale jest dostarczony z Anacondą, co oznacza łatwość dostepu.
\item Argparse - to biblioteka w języku Python, która służy do parsowania argumentów linii poleceń. Argparse jest częściową standardowej biblioteki Pythona co oznacza że jest wbudowany w standardową instalacje Anacondy.
\item Pytest to biblioteka w języku Python, która służy do testowania kodu żródłowego. Umożliwia łatwe i elastyczne pisanie testów. Pytest nie jest wbudowany w standardową instalację Pythona ani w dystrybucji pakietów Anaconda, ale można ją zainstalować za pomocą menadzera pakietów pip.
\item Os to biblioteka standardowa w języku Python, która zapewnia interfejs do operacji na systemie operacyjnym (np. dostęp do plików, zarządzanie procesami, zmiana katalogu roboczego itp.). 
\item Tkinter to biblioteka graficzna dla języka programowania Python. Biblioteka ta umożliwia tworzenie interfejsów graficznych użytkownika (GUI) dla programów Python. Tkinter jest dostępny w standardowej bibliotece Pythona i jest łatwo dostępny na większości platform.

\end{enumerate}

Należy również pobrać plik tekstowy o nazwie "wsp\_{}inp.txt", który znajduje się na zdalnym repozytorium GitHub pod linkiem: \href{https://github.com/Sawoboh/Informatyka_Projek_1.git}{https://github.com/Sawoboh/Informatyka\_{}Projek\_{}1.git} da on możliwość wczytania i wykonania transformacji z zawierających danych. Nastepnie zapisze te dane do pliku wynikowego.  