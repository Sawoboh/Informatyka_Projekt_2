\documentclass[10pt,a4paper]{article}

% ------------------------- PREAMBUŁA -------------------
\usepackage{array}
\usepackage{multicol}       % \multicolumn{num_cols}{alignment}{contents}
\usepackage{multirow} 
\usepackage[polish]{babel}
\usepackage{graphicx}
\usepackage{amsmath, amsfonts, amssymb}
\usepackage[utf8]{inputenc}
\usepackage[T1]{fontenc}
\usepackage{multicol}
\usepackage{booktabs}
\usepackage{hyperref} 
\usepackage[figurename=Fig., tablename=Tab.]{caption}
\usepackage[top=2.5cm, bottom=2.5cm, left=2.5cm, right=2.5cm]{geometry}
\RequirePackage{kantlipsum} % English kantian-style lipsum (wypełniacze tekstu)
\RequirePackage{lipsum}
\usepackage{xcolor}                         % definiowanie kolorów
\newcommand{\myRed}[1]{\textcolor{red}{#1}} % wykorzystanie makro \myRed{tekst}
\definecolor{GIKorange}{HTML}{C06000}       % definiowanie własnego koloru
\newcommand{\myOrange}[1]{\textcolor{GIKorange}{#1}} % wykorzystanie makro \myOrange{tekst}

\definecolor{pw_sloneczny}{RGB}{254, 213, 66}
\definecolor{pw_morelowy}{RGB}{234, 124, 90}
\definecolor{pw_mietowy}{RGB}{106, 186, 156}
\definecolor{pw_sliwkowy}{RGB}{150, 95, 119}
\definecolor{pw_szafirowy}{RGB}{120, 150, 207}
\definecolor{pw_wrzosowy}{RGB}{180, 160, 170}
\definecolor{pw_grafitowy}{RGB}{60, 60, 76}
\definecolor{pw_mokka}{RGB}{100, 90, 90}

\usepackage{graphicx}       % wstawianie grafiki
\DeclareGraphicsExtensions{.pdf,.eps,.png,.jpg}

\usepackage{float}          % to fix table/figure location \begin{figure}[ht]
\usepackage{subfloat}       % subfigure options
\usepackage{caption}            
\usepackage{subcaption}
\usepackage{array}
\usepackage{multicol}       % \multicolumn{num_cols}{alignment}{contents}
\usepackage{multirow}       % \multirow{''num_rows''}{''width''}{''contents''} {* for the width means the content's natural width}
\usepackage{longtable}  % Multi-page tables
\usepackage{booktabs}       % tabele (toprule/midrule/bottomrule)
\usepackage{tabularx}   % to wrap words (here use in enviroment condition)
%--------------------------------
% CONDITIONS: A new enviroments  to explain parameters in equations
%--------------------------------
\newenvironment{conditions*}
{\par\vspace{\abovedisplayskip}\noindent
	\tabularx{\columnwidth}{>{$}l<{$} @{${}: {}$} >{\raggedright\arraybackslash}X}}
{\endtabularx\par\vspace{\belowdisplayskip}}


\hypersetup{hidelinks,
	bookmarks=true,
	bookmarksopen=true,
	colorlinks=true,            % hyperlinks will be coloured
	linkcolor=pw_mietowy,            % hyperlink text will be green
	linkbordercolor=black,      % hyperlink border will be red
	citebordercolor=black,      % color of links to bibliography
	citecolor=black,
	filebordercolor=black,      % color of file links
	urlbordercolor=black,       % color of external links
	filecolor=black,
	menucolor=black,
	urlcolor=black,
	backref= page,              % page, section(s)
	pagebackref = true,         % to show number of page with citations backref=none
	plainpages=true,            % false
	pdfpagelabels,
	hypertexnames=true,         % false
	linktocpage}


% ----- modyfikacja pakietu hyperref aby jedynie podkreślał link
%\makeatletter
%\Hy@AtBeginDocument{%
	%\def\@pdfborder{0 0 1}% Overrides border definition set with colorlinks=true
	%\def\@pdfborderstyle{/S/U/W 1}% Overrides border style set with colorlinks=true
	%                              % Hyperlink border style will be underline of width 1pt
	%}
%\makeatother

%--------------------------------------
% Bibiography specification ver PL
%--------------------------------------
\RequirePackage[natbibapa]{apacite}	    % bibliography natbib - natbibapa loads the natbib package for the citation commands
\bibliographystyle{apacite}
%\nocite{*} % uzyj aby uwzględnić wszystkie odniesienia w bazie danych .bib (bez tego jedynie cytowane pozycje)
\renewcommand{\BBAA}{i}  % between authors in parenthetical cites and ref. list
\renewcommand{\BBAB}{i}  % between authors in in-text citation
\renewcommand{\BAnd}{i}  % for ``Ed. \& Trans.'' in ref. list
\renewcommand{\BOthers}{i in}
\renewcommand{\doiprefix}{DOI:}
\urlstyle{APACrm}



% ------------------------------------------------------------------------------------------------------------
% --- Wstawianie fragmentów kodu
% ------------------------------------------------------------------------------------------------------------
%--------------------------------------
% Code  listings and verbatim ()
%--------------------------------------
\usepackage{courier}               % to keep bold for \ttfamily
\usepackage{verbatim}              % environment which looks for the exact string 
\usepackage{pmboxdraw}             % alows to use special unicode character U + 20000
\RequirePackage{listings}              % Code listings
\RequirePackage{ucs}                   % tosupportencodingutf8x.
\renewcommand{\lstlistingname}{Kod}    % put the name of caption 
% general option of listing
\lstset{aboveskip       = 1.mm,
	belowskip       = 1.mm,
	frame           = single,
	numbers         = left,
	firstnumber     = 1,
	xleftmargin     = 2em,
	numberstyle     = {\footnotesize},
	basicstyle      ={\footnotesize\ttfamily}, % \ttfamily
	breaklines      = true,
	backgroundcolor = \color{pw_grafitowy!1!white},
	commentstyle    = \color{pw_grafitowy!85!white}\textit,
	stringstyle     = \color{pw_grafitowy},
	keywordstyle    = \bfseries,
	otherkeywords   = {def, return, assert, with, try, exept, Lambda, Map, Fiter, yield, nonlocal, bytes, Ellipsis, NotImplemented, None, setdefault, read, readline, readlines, write, writeline, close, self, SyntaxError, ZeroDivisionError, SyntaxError, IndentationError, IndexError, AssertionError, KeyError, TypeError, NameError, AttributeError, ValueError, RuntimeError,False, True},                
	breaklines      = true,
	postbreak=\mbox{\textcolor{red}{\rotatebox[origin=c]{180}{$\Lsh$}}\space}, % break line for long text
	inputencoding   = utf8x, % incorporate polish characters
	extendedchars   = \true,
	literate={ą}{{\k{a}}}1
	{Ą}{{\k{A}}}1
	{ę}{{\k{e}}}1
	{Ę}{{\k{E}}}1
	{ó}{{\'o}}1
	{Ó}{{\'O}}1
	{ś}{{\'s}}1
	{Ś}{{\'S}}1
	{ł}{{\l{}}}1
	{Ł}{{\L{}}}1
	{ż}{{\.z}}1
	{Ż}{{\.Z}}1
	{ź}{{\'z}}1
	{Ź}{{\'Z}}1
	{ć}{{\'c}}1
	{Ć}{{\'C}}1
	{ń}{{\'n}}1
	{Ń}{{\'N}}1
}


   % ścieżka względna do katalogu z ustawieniami


% METADATA
% DOCUMENT METADATA
\newcommand{\logoGIK}{settings/WGiK-znak.png}
\newcommand{\authorName}{Adrian Maksymiuk  \\ grupa II, Numery Indeksu: 319345 \\ Dawid Jundo\\ grupa II, Numery Indeksu: 319328}

\newcommand{\titeReport}{Transformacje miedzy różnymi układami geodezyjnymi} 
\newcommand{\titleLecture}{Informatyka Geodezyjna \\ sem. IV, ćwiczenia, rok akad. 2022-2023} 
\newcommand{\kind}{report}
\newcommand{\mymail}{\href{mailto:01169882@pw.edu.pl}{01169882@pw.edu.pl} lub \href{mailto:01169863@pw.edu.pl}{01169863@pw.edu.pl}}
\newcommand{\filledsquare}{\textcolor{black}{\rlap{$\blacksquare$}\rule[0.1ex]{1.8ex}{1.8ex}}}
\newcommand{\supervisor}{....}
\newcommand{\gikweb}{\href{www.gik.pw.edu.pl}{www.gik.pw.edu.pl}}
\newcommand{\institut}{Zakład Geodezji Wyższej i Astronomii}
\newcommand{\faculty}{Wydział Geodezji i Kartografii}
\newcommand{\university}{Politechnika Warszawska}
\newcommand{\city}{Warszawa}
\newcommand{\thisyear}{2023}
%\date{}
% PDF METADATA
\pdfinfo
{
	/Title       (GIK PW)
	/Creator     (TeX)
	/Author      (Imię Nazwisko)
}

\begin{document}

	\begin{center} 
		\rule{\textwidth}{.5pt} \\
		\vspace{1.0cm}
		\includegraphics[width=.4\paperwidth]{\logoGIK}
		\vspace{0.5cm} \\
		\Large \textsc{\titeReport}
		\vspace{0.5cm} \\  
		\large \textsc{\titleLecture}
		\vspace{0.5cm}\\
		\textsc{\authorName}  \\
		\mymail 
		\vspace{0.5cm} \\
		\textsc{\faculty}\\ \textsc{\university}  \\ \institut\\
		\city, \today
	\end{center} 
	\rule{\textwidth}{1.5pt}
	
	
	\tableofcontents 								% wyświetla spis treści
	%\addcontentsline{to}{chapter}{Spis treści} 	% dodaje pozycję do spisu treści
	% \listoffigures  								% wyświetla spis rysunków
	%\addcontentsline{toc}{chapter}{Lista rysunków} % dodaje pozycję do spisu treści
	% \listoftables 									% wyświetla spis rysunków
	%\addcontentsline{toc}{chapter}{lista tabel}	% dodaje pozycję do spisu treści
	\newpage
	\input{content/1-Cel_ćwiczenia1.tex} 
	\section{Wykorzystane narzedzia i materiały potrzebne do replikacji ćwiczenia}

\subsection{Wybrany język programowania i interpreter Spyder}

\begin{itemize}
\item Python - język programowania, w którym napisany jest skrypt ćwiczenia.
\item Spyder - jest to środowisko programistyczne dla języka Python, które zawiera edytor kodu, interpreter, konsolę i wiele innych funkcjonalności.
\item Najlepiej pobrać Spydera za pośrednictwem anacondy, która ma domyślie zainstalowane środowisko programistyczne Spyder \href{https://www.anaconda.com/download}{www.anaconda.com/download}	\citep{Anaconda}
\end{itemize}

\subsection{System operacyjny}

Ten skrypt został napisany w systemie operacyjnym Microsoft (Windows 10 oraz Windows 11).

\subsection{Potrzebne biblioteki i pliki}

Do wykonania ćwiczenia należy użyć następujących bibliotek:
\begin{enumerate}
\item Numpy - to biblioteka w języku Python, która służy do obliczeń numerycznych i analizy danych. Numpy dostarcza wiele narzędzi do pracy z wielowymiarowymi tablicami danych oraz narzędzi do wykonywania operacji matematycznych i statystycznych na tych tablicach. Numpy nie jest wbudowany w Pythona, ale jest dostarczony z Anacondą, co oznacza łatwość dostepu.
\item Argparse - to biblioteka w języku Python, która służy do parsowania argumentów linii poleceń. Argparse jest częściową standardowej biblioteki Pythona co oznacza że jest wbudowany w standardową instalacje Anacondy.
\item Pytest to biblioteka w języku Python, która służy do testowania kodu żródłowego. Umożliwia łatwe i elastyczne pisanie testów. Pytest nie jest wbudowany w standardową instalację Pythona ani w dystrybucji pakietów Anaconda, ale można ją zainstalować za pomocą menadzera pakietów pip.
\item Os to biblioteka standardowa w języku Python, która zapewnia interfejs do operacji na systemie operacyjnym (np. dostęp do plików, zarządzanie procesami, zmiana katalogu roboczego itp.). 
\item Tkinter to biblioteka graficzna dla języka programowania Python. Biblioteka ta umożliwia tworzenie interfejsów graficznych użytkownika (GUI) dla programów Python. Tkinter jest dostępny w standardowej bibliotece Pythona i jest łatwo dostępny na większości platform.

\end{enumerate}

Należy również pobrać plik tekstowy o nazwie "wsp\_{}inp.txt", który znajduje się na zdalnym repozytorium GitHub pod linkiem: \href{https://github.com/Sawoboh/Informatyka_Projek_1.git}{https://github.com/Sawoboh/Informatyka\_{}Projek\_{}1.git} da on możliwość wczytania i wykonania transformacji z zawierających danych. Nastepnie zapisze te dane do pliku wynikowego.   
	\input{content/3-Przebieg ćwiczenia1.tex}
	
	\section{Podsumowanie}

\subsection{Rezultaty}
Link do zdalnego repozytorium GitHub: \href{https://github.com/Sawoboh/Informatyka_Projek_1.git}{https://github.com/Sawoboh/Informatyka\_{}Projek\_{}1.git} \\
Znajduje sie na nim pliki o nazwie:
\begin{itemize}
\item Transformacje\_{}Projekt.py - główny plik w którym znajdują się transformacje oraz wywołanie przykładowego pliku txt wraz z zapisym do do pliku o nazwie Wyniki transformacji.txt.
\item wsp\_{}inp.txt - zawierają przykładowe dane, które możemy przliczyć. 
\item Kalkulator.py -  Cztery plik importujący biblioteke argparse. Za pomocą wiersza poleceń (cmd) należy  podać dane do obliczeń.
\item Testy\_{}dla\_{}funkcji.py - plik importujący biblioteke pytest. Za pomocą wiersza poleceń (cmd) podać dane do obliczenia.
\item TKinter.py - plik wywołuje kalkulator graficzny i okno wynikowe. 
\end{itemize}

\subsection{Umiejętności nabyte}
\begin{itemize}
	\item Sprawne pisanie plików tekstowych w latex w celu nauki skorzystliśmy z książki \citep{Borkowski.Przybylski2015}
	\item Pisanie kodu obiektowego w Pythonie 
	\item Posługiwanie się bibliotekami takimi jak: argparse, pytest, tkinter, os, numpy \citep{argparse} \citep{argparse2}
	\item Praca zespołowa z wykorzystaniem platformy Github
	\item Poprawienie jakości i przyśpieszenie pisania kodu w Pythonie
\end{itemize}

\subsection{Spostrzerzenia, probelmy i ich rozwizania:}
Spostrzerzenia:
\begin{itemize}
\item Wraz ze wzrostem ilości czasu poświęconego na projekt, zauważano coraz wiecej luk w funkcjach, które usprawniono.
\item Cały czas program nie jest kompletny w 100\%. Zawsze się znajdzie nowy pomysł który można zaimplementować.
\end{itemize}

\begin{table}[h!]
	\centering
	\begin{tabular}{|p{7cm}|p{7cm}|}
		\hline
		 Problem  &  Rozwizanie \\
		\hline
		Brak spójności w długości liczb w tabeli przez co plik wynikowy nie wyglądałby schludnie & Dodanie funkcji na dodanie spacji w stringu opisany w rozdziale 3.7  \\ \hline
		Pojawianie się błedu w funkcji PL2000 i PL 1992 dla ($\phi$, $\lambda$) nie leżącego na terenie Polski & Dodane warunku na ($\phi$, $\lambda$) w którym jeśli warunek nie jest spełniony zapisuje w notatniku myslinik (-)\\ \hline
		Dopisywanie do istniejącego pliku wyników & Wykorzystano biblioteke os do rozpoznania czy dany plik istnieje i podanie odpowiednich komend w przypadku gdy plik zostaje stworzony i do pliku zostają dopisane wyniki\\ \hline
		Nieumijętność korzystania z biblioteki argparse & Doptytanie się na zajęciach informatyki geodezyjnej prowadzącego o wytłumaczenie jak korzystać z biblioteki \\ \hline
		Nieumijętność korzystania z portala Github & Skorzystanie z materiałów uspodtępnionych przez prowadzącego zajecia informatyka geodezyjna\\ \hline
		Brak możliości zresetowania tablicy z wynikami w pliku pytest &  \\ \hline
		
	\end{tabular}
\end{table} 
	
	
	\bibliography{bib}

	
	

	
	
	
\end{document}

